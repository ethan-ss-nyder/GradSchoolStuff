\documentclass[11pt]{article}

\newcommand{\school}{Michigan State University}
\newcommand{\schoolabbr}{MSU}
\newcommand{\dept}{Department of Physics and Astronomy}

\usepackage[margin=1.51in]{geometry}

\title{\vspace{-5em}Personal Statement\\
	\large \dept{} at \school{}}

\author{Ethan Snyder}
\date{}

\begin{document}
\maketitle
\subsubsection*{Making a community at WMU}
When I first started undergrad at Western Michigan University, the world was recovering from COVID, and for the few in-person courses I had, I commuted 30 minutes from home. I felt detached and isolated.

Sophomore year, I moved out to live near campus. I made a friend, a fellow physics major, and he and I would study together. We sought out other physics majors to study together and to make a support system, but found that other physics majors were as detached as I had been. We also found that WMU's Physics Club had not been active in years.

We spent the following summer deep-cleaning the physics commons, and in the fall of 2023, he and I registered Physics Club as an official Registered Student Organization (RSO), opened a bank account in its name, and I started attending biweekly meetings to secure funding for potential events. We agreed that as co-founders, we would be co-presidents, and I assumed the extra tasks of being treasurer.

It wasn't until 2024's solar eclipse trip that Physics Club gained traction. I called around to universities and cities for permission to use their land in the path of totality, and we invited the Women in STEM organization at WMU to include them in the event. In the end, we were able to advertise enough to fill a bus, and it was an incredible experience. While Physics Club does many smaller events (such as astronomy outings and movie nights), this event, more than any other, exercised my ability to organize, lead, and execute plans.

Fall 2024, the semester in which I am writing this, has seen a boom in Physics Club members, and we're now in the process of partnering with Women in STEM and the Society of Hispanic Professional Engineers to plan a tour of Fermilab. I'm also serving as a link for underclassmen to professors and graduate students, introducing undergraduates to research opportunities and creating a stronger sense of community within the department. I also provide tutoring and guidance to the underclassmen through courses I've already taken, strengthening my understanding of old course material. The extra guidance, whether in coursework or in considering research opportunities, is a luxury I wish I'd had as an underclassman, and I'm proud to be facilitating this for incoming physics students.

The physics commons room can now be somewhat loud during the day. We now have members from these RSOs we've collaborated with, and the diversity of students, voices, and backgrounds in the physics commons is refreshing in contrast to the small city in the corn fields where I grew up. I'm proud to have been a part of founding Physics Club and providing a supportive, safe space for all those interested in physics. It's immensely rewarding to finally be a part of that community that I had sought out when I was a freshman.
%%%%%%%%%%%%%%%%%%%%%%%%%%%%%%%%%%%%%%%%%%%%%%%%%%%%%%%%%%%%%%%%%%%%%%%%%%%%%%%%%%%%%%%%%%%%%%%%%%%%%%%
\subsubsection*{Persistence in the Face of Adversity}
I am a first generation college student. I grew up as the youngest of four boys in a single income household, always hovering just above the poverty line, and I am the only of the brothers to attempt a college degree. My mom has always said \textit{I don't know who you got that brain from}, but I figure I got it from her. She just never had the opportunity to commit four years of her life to college.

At the tail end of high school, my parents told me they were getting divorced. It was the type of situation where I was not shocked or overly distraught --- that is until the divorce lost its civility and my family cut contact with my dad. One of the consequences of this was my new, sudden awareness of financial instability, and this was a non-negligible factor in my commitment to condensed matter physics. Not only am I investing in my own education, but myself and my mom's future and stability.

While I was adjusting to college life, I found that I didn't have much time or energy to be present on campus. In addition to being a full time student and a part time worker, I now had to make time to appear in court and help my family clean and move out of our childhood home. In that span of time --- freshman, sophomore year --- I had to make significant change to my life and all at once transition from teenager to adult. The effects of this time period are apparent on my transcript, but also apparent is my resiliency and perseverance: my grades improved, I began taking summer courses, and I began taking more credits. I got through this period with a deep understanding of the fundamentals that enabled me to excel in later courses.

Once I emerged from this low point in my life, I hit the ground running. Midway through my sophomore year, my family issues resolved and I finally felt it was appropriate to move to an off campus apartment, and I got serious about my education. I declared a math major on top of my physics major, and a computer science minor in addition to my astronomy minor. The next semester, I began working with Dr.\@ O'Hara of WMU's Physics Department, and I continue that research to this day.

I was determined to not let this experience hold me back, and I believe I have more than made up for that lost time and energy with the motivation I have to learn and my dedication to progress, to succeed. While neither myself nor my family have fully recovered from this event, I'm proud that we made it through and I'm proud to have laid a strong foundation of knowledge during those times. Looking forward, I'm comforted having proven to myself that I can handle anything, whether that be trauma in my family and personal life or the 18-credit-hour semesters that followed. This same persistence and motivation guarantees my full effort towards succeeding at \school{}.
\end{document}
