\documentclass[11pt]{article}

\usepackage{todonotes}
\newcommand{\school}{University of Illinois Urbana-Champaign}
\newcommand{\schoolabbr}{Illinois}
\newcommand{\dept}{Department of Physics}

%\usepackage[margin=1.5in]{geometry}

\title{\vspace{-5em}Academic Statement\\
	\large \dept{} at \school{}}

\author{Ethan Snyder}
\date{}

\begin{document}
\maketitle
\subsubsection*{My Academic Experience}
As I'm sure is the case with any condensed matter physicist, I haven't always known I would pursue condensed matter physics. Telescopes inspire children to astronomy, microscopes inspire children to biology, and condensed matter physics is left to allure the occasional, unsuspecting undergraduate student like myself.

As a freshman at Western Michigan University, I was certain I wanted to study astrophysics in graduate school. However, I grew to be fascinated by the physics of the small and the physics governing the technology of the modern world. After assisting with semiconductor defect research for over a year, I knew I wanted to pursue condensed matter physics in graduate school. This research affirmed my deep appreciation for computers, and ever since I've been certain that I want to work with simulations, models, and work in fields applicable to computers, either quantum or classical. This is why I'm motivated to pursue condensed matter physics and quantum information science in graduate school --- I want to help develop the technology of the future. Graduate school is my next step, as it will allow me to further develop these interests, setting me up for my goal of researching semiconductors or quantum computing in the industry or as a professor.\ \schoolabbr{}, given the diversity and depth of research in these fields, is the perfect place for me to accomplish this goal.

In my four years at WMU, I've been working towards physics and mathematics majors with computer science and astronomy minors. This wasn't always the plan, as I realized halfway through my undergraduate experience that I enjoyed learning about and working with computers too much to not incorporate computer science into my education. I also realized that I could benefit from the stronger math skills that came with math as a major instead of a minor, and I now find concepts like topology and group theory in physics more approachable than they would have been otherwise.

\textbf{By far my most extensive research experience is my research on semiconductor deep acceptor center defects with Dr.\@ O'Hara of the WMU Physics Department}. The goal of this research is to calculate formation energies and model structural distortions of defects in gallium and aluminum rich AlGaAs alloys to corroborate and explain experimental findings. This research has familiarized me with solid state physics, semiconductors, crystal lattice structures, density functional theory, VASP, different functionals like PBE and HSE, alloying techniques like the virtual crystal approximation (VCA), explicit quasi-random structures, and much more. I've also learned countless things about how to interface with computer clusters, how to move files around, how to analyze data, and how to write scripts to do all of this efficiently.

Additionally, in the spring, I will be beginning a semester project with Dr.\@ Korista of the Physics Department modeling stellar evolution in different classes of stars. This will provide me with a diversity of experiences with computational physics and data analysis across fields and with many programming languages.

During my time at WMU, I've learned to operate and collect data from the Tandem Van de Graaff Accelerator in Rood Hall and have occasionally worked with the research team that regularly operates it. The peak of my involvement was in the summer of 2024, when the department received cable samples from NASA with the request that we measure radiation tolerances. As of this semester, I have also been spending time with the physics department's machine shop specialist learning to manufacture simple parts, learning machine shop etiquette and safety, and how to communicate jobs effectively with machinists. Although my skills are best suited for theoretical, computational work, I believe my familiarity with the lab setting and part manufacturing has made me a well-rounded physicist and better able to help in experimental settings when necessary.
\pagebreak

\phantom{}
\vspace{-5em}
\subsubsection*{Why I choose the \school{}}
\schoolabbr{} would provide me ample opportunity to build upon the knowledge and skills I've gained in computational condensed matter physics research, while also providing opportunities to explore a wide range of related topics and subfields.\ \schoolabbr{} is the perfect place to get the education, experience, and expertise I need to go forward into a career, onto the frontier of research. Of all the Physics Department's research, the Ceperley, Bradlyn, and Fernandes groups stand out to me the most.

I'm most interested in the Ceperley group, as I'll have over a year of continuous experience in condensed matter research by the time I graduate, and I'm interested in pushing my current knowledge to the absolute fundamentals of many-body systems. I have some understanding of quantum many-body systems, electronic structures of solid state materials, and computational research techniques, including interfacing and running calculations on a computer cluster, which should provide a good starting place for working with Dr.\@ Ceperley and understanding his research.

Dr.\@ Fernandes's group most closely aligns with my current area of research, and thus would be the group I'm most capable of contributing to. In particular, his recent work involving doping, disorder, and electronic structures of materials is familiar from my research with semiconductors. I also love the idea of working with experimental groups while also staying in the realm of computational research. Overall, Dr.\@ Fernandes's work is very compatible with my experience, and I'd be interested in working with him in the future.

Leaning more into my interest of topology and abstract algebra, I can also see myself working with Dr.\@ Bradlyn's research team. His research seems to be a steep learning curve, but I believe my higher-level math skills give me a good start on understanding topology and its applications to condensed matter. The skills I've learned in undergrad have equipped me to take on such large tasks, and I'm interested in elevating my understanding of condensed matter physics to the exotic phenomena that the Bradlyn group studies.

I'm eager to apply all that I've learned towards the fascinating research at \schoolabbr{}, particularly in areas that align with my thorough computational experience with condensed matter physics. I look forward to the opportunity to learn from and contribute to your research teams, and I am confident that I'd make a capable, committed researcher and condensed matter physicist. Thank you for your time and for considering my application.
\end{document}