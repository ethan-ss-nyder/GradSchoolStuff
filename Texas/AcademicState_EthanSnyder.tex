\documentclass[11pt]{article}

\usepackage{todonotes}
\newcommand{\school}{University of Texas at Austin}
\newcommand{\schoolabbr}{UT Austin}
\newcommand{\dept}{Department of Physics}

%\usepackage[margin=1.5in]{geometry}

\title{\vspace{-5em}Academic Statement\\
	\large \dept{} at \school{}}

\author{Ethan Snyder}
\date{}

\begin{document}
\maketitle
\subsubsection*{My Academic Experience}
As I'm sure is the case with any condensed matter physicist, I haven't always known I would pursue condensed matter physics. Telescopes inspire children to astronomy, microscopes inspire children to biology, and condensed matter physics is left to allure the occasional, unsuspecting undergraduate student like myself.

As a freshman at Western Michigan University, I was certain I wanted to study astrophysics in graduate school. However, I grew to be fascinated by the physics of the small and the physics governing the technology of the modern world. After assisting with semiconductor defect research for over a year, I knew I wanted to pursue condensed matter physics in graduate school. This research affirmed my deep appreciation for computers, and ever since I've been certain that I want to work with simulations, models, and work in fields applicable to computers, either quantum or classical. This is why I'm motivated to pursue condensed matter physics and quantum information science in graduate school. Graduate school is my next step, as it will allow me to further develop these interests, setting me up for my goal of researching semiconductors or quantum computing in the industry or as a professor.\ \school{}, given the diversity and depth of research in these fields, is the perfect place for me to accomplish this goal.

In my four years at WMU, I've been working towards physics and mathematics majors with computer science and astronomy minors. This wasn't always the goal, as I realized halfway through my undergraduate experience that I enjoyed learning about and working with computers too much to not incorporate computer science into my education. I also realized that I could benefit from the stronger math skills that came with math as a major instead of a minor, and I now find concepts like topology and group theory in physics more approachable than they would have been otherwise.

Despite the hefty course load, I've been a dean's list student, been awarded a research scholarship through WMU's Undergraduate Research, Ingenuity and Creativity Summer Scholarship program (URICS), and most notably, I've received the Paul Rood Physics Scholarship from the Physics Department acknowledging exceptional performance within the department. I have certainly laid the foundations to be successful in computational physics research through both coursework and research experience, and I believe these accolades prove my capability.

\textbf{By far my most extensive research experience is my research on semiconductor deep acceptor center defects with Dr.\@ O'Hara of the Physics Department}, which, at the time of writing, I currently run on MSU's HPCC through ICER.\ The goal of this research is to calculate formation energies and model structural distortions of defects in gallium and aluminum rich AlGaAs alloys to corroborate and explain experimental findings. This research has familiarized me with solid state physics, semiconductors, crystal lattice structures, density functional theory, VASP, different functionals like PBE and HSE, alloying techniques like the virtual crystal approximation (VCA), explicit quasi-random structures, and much more. I've also learned countless things about how to interface with computer clusters, how to move files around, how to analyze data, and how to write scripts to do all of this efficiently. I've attached a digital copy of a poster I presented at WMU's Lee Honors College as part of the previously mentioned URICS scholarship that summarizes much of my prep work and alludes to the work I'm currently doing.

Additionally, in the spring, I will be beginning a semester project with Dr.\@ Korista of the Physics Department modeling stellar evolution in different classes of stars. This will provide me with a diversity of experiences with computational physics and data analysis across fields and with many programming languages.

During my time at WMU, I've learned to operate and collect data from the Tandem Van de Graaff Accelerator in Rood Hall and have occasionally worked with the research team that regularly operates it. The peak of my involvement was in the summer of 2024, when the department received cable samples from NASA with the request that we measure radiation tolerances. As of this semester, I have also been spending time with the physics department's machine shop specialist learning to manufacture simple parts, learning machine shop etiquette and safety, and how to communicate jobs effectively with machinists. Although my skills are best suited for theoretical, computational work, I believe my familiarity with the lab setting and part manufacturing has made me a well-rounded physicist and better able to help in experimental settings when necessary.
\pagebreak

\phantom{}
\vspace{-5em}
\subsubsection*{Why I choose \school{}}
\schoolabbr{} would provide me ample opportunity to build upon my knowledge and skills I've gained in computational condensed matter physics research, while also providing opportunities to explore a wide range of related topics. My confidence in navigating and running jobs on MSU's HPCC is a directly transferable skill that would enable me to quicker contribute to theoretical groups such as those of Dr.\@ Maghrebi and Dr.\@ Dykman.\ \schoolabbr{} is the perfect place to get the education, experience, and expertise I need to go forward into a career, onto the frontier of research. The Maghrebi, Dykman, and Becker groups stand out to me the most.

I'm most interested in the Maghrebi group, as I'll have over a year of continuous experience in condensed matter research by the time I graduate, and I'm interested in pushing my knowledge to non-equilibrium dynamics and many-body systems. I will have some understanding of many-body systems and computational research techniques, and I'm excited about the possibility of contributing to this team and further developing my expertise in this field.

Leaning more into my interest of quantum information science and quantum computing, I can see myself working with Dr.\@ Dykman's research team. His work involving qubits and qubit frequency noise in particular is interesting, and I am confident that my thorough computational research experience will allow me to make meaningful contributions to this team. His recent work on electrons in helium and nanomechanics is also very interesting to me, and I'd be interested to be a part of that research in the future. Overall, Dr.\@ Dykman's research seems to be a steep learning curve, but I know the skills I've learned in undergrad have equipped me to take on such large tasks.

Dr.\@ Becker's group really appeals to my interest in quantum information science and quantum computing, and I believe my extensive background in solid-state physics and atomic dopants and defects would provide me a great starting point in understanding and contributing to this team. My familiarity with the lab setting and data analysis would enable me to learn from and contribute to the group's work.

I'm eager to apply all that I've learned towards the fascinating research at \schoolabbr{}, particularly in areas that align with my deep interest in quantum computing and my thorough experience with condensed matter physics. I look forward to the opportunity to learn from and contribute to your research teams, and I am confident that I'd make a capable, committed researcher and condensed matter physicist. Thank you for your time and for considering my application.
\end{document}
