\documentclass[11pt]{article}

\usepackage{todonotes}
\newcommand{\school}{University of Colorado Boulder}
\newcommand{\schoolabbr}{CU Boulder}
\newcommand{\dept}{Department of Physics}

\usepackage[margin=1.1in]{geometry}

\title{\vspace{-5em}Personal Statement\\
	\large \dept{} at \school{}}

\author{Ethan Snyder}
\date{}

\begin{document}
\maketitle
\subsubsection*{My Academic Experience}
As I'm sure is the case with any condensed matter physicist, I haven't always known I would pursue condensed matter physics. Telescopes inspire children to astronomy, microscopes inspire children to biology, and condensed matter physics is left to allure the occasional, unsuspecting undergraduate student like myself.

During my time at Western Michigan University, I grew to be fascinated by the physics of the small and the physics governing the technology of the modern world. After assisting with semiconductor defect research for over a year, I knew I wanted to pursue condensed matter physics in graduate school. This research affirmed my deep appreciation for computers, and ever since I've been certain that I want to work with simulations, models, and work in fields applicable to computers, either quantum or classical. This is why I'm motivated to pursue condensed matter physics and quantum information science in graduate school. Graduate school is my next step, as it will allow me to further develop these interests, setting me up for my goal of researching semiconductors or quantum computing in the industry or as a professor. The \school{}, given the diversity and depth of research in these fields, is the perfect place for me to accomplish this goal.

In my four years at WMU, I've been working towards physics and mathematics majors with computer science and astronomy minors. This wasn't always the goal, as I realized halfway through my undergraduate experience that I enjoyed learning about and working with computers too much to not incorporate computer science into my education. I also realized that I could benefit from the stronger math skills that came with math as a major instead of a minor, and I now find concepts like topology and group theory in physics more approachable than they would have been otherwise. The workload, time management, and work ethic I learned in these four years ensure I can handle the responsibilities of graduate school.

Despite the hefty course load, I've been a dean's list student, been awarded a research scholarship through WMU's Undergraduate Research, Ingenuity and Creativity Summer Scholarship program (URICS), and most notably, I've received the Paul Rood Physics Scholarship from the Physics Department acknowledging exceptional performance within the department. I have certainly laid the foundations to be successful in computational physics research through both coursework and research experience, and I believe these accolades prove my capability.

\textbf{By far my most extensive research experience is my research on semiconductor deep acceptor center defects with Dr.\@ O'Hara of the Physics Department}, which I currently run on Michigan State University's High-Performance Computing Cluster (HPCC).\ The goal of this research is to calculate formation energies and model structural distortions of defects in gallium and aluminum rich AlGaAs alloys to corroborate and explain experimental findings. This research has familiarized me with solid state physics, semiconductors, crystal lattice structures, density functional theory, VASP, different functionals like PBE and HSE, alloying techniques like the virtual crystal approximation (VCA), explicit quasi-random structures, and much more. I've also learned countless things about how to interface with computer clusters, how to move files around, how to analyze data, and how to write scripts to do all of this efficiently. I've attached a digital copy of a poster from my URICS scholarship work that I presented at WMU's Lee Honors College, summarizing my preparatory work and alluding to my current research.

Additionally, in the spring, I will be beginning a semester project with Dr.\@ Korista of WMU's Physics Department modeling stellar evolution in different classes of stars. This will provide me with a diversity of experiences with computational physics and data analysis across fields and with many programming languages.

During my time at WMU, I've learned to operate and collect data from the Tandem Van de Graaff Accelerator in Rood Hall and have occasionally worked with the research team that regularly operates it. The peak of my involvement was in the summer of 2024, when the department received cable samples from NASA with the request that we measure radiation tolerances. As of this semester, I have also been spending time with the physics department's machine shop specialist learning to manufacture simple parts, learning machine shop etiquette and safety, and how to communicate jobs effectively with machinists. Although my skills are best suited for theoretical, computational work, I believe my familiarity with the lab setting and part manufacturing has made me a well-rounded physicist and better able to help in experimental settings when necessary.

\subsubsection*{Why I choose the \school{}}
\schoolabbr{} would provide me with ample opportunity to build upon my knowledge and skills I've gained in computational condensed matter physics research, while also providing opportunities to explore a wide range of related topics. In particular, the collaborative environment between \schoolabbr{} and NIST and the resources available to the UC Boulder Physics Department appeals to me. Also, my confidence in navigating large computer clusters and running jobs on them is a directly transferable skill that would enable me to quicker contribute to theoretical, computational groups such as those of Dr.\@ Lucas. Along with Dr.\@ Lucas' group, I can also see myself contributing to Dr.\@ Lehnert's group.

I'm most interested in the Lucas group, as I'll have over a year of continuous experience in condensed matter physics research by the time I graduate, and I'm very interested in expanding my work and knowledge into research leaning more into quantum information science. Dr.\@ Lucas' recent work on many-body quantum systems, quantum computing/qubits, and other quantum phenomena, is exactly the type of work I would like to participate in. I believe my background in computational physics and knowledge in condensed matter physics would enable me to contribute to this work and provide a stable starting point as I learn.

On the experimental end of things, I can also see myself working with Dr.\@ Lehnert. Although my background is better suited for computational, theoretical applications, working with Dr.\@ Lehnert would allow me a greater chance in expanding into the industry in the future. The applications of his work to quantum computing and its roots in condensed matter physics makes it both approachable and interesting to me. My experience with experimental research, and my computational skills geared towards data analysis, would make me a useful addition to Dr.\@ Lehnert's team.
\pagebreak

\subsection*{Adversity and My Commitment to an Inclusive, Diverse Environment}
As a first-generation college student from a family of six living near the poverty line, I was never encouraged to excel --- my parents just wanted me to pass. I owe my academic success to a supportive school district and teachers who saw my potential early on, despite my behavioral issues in elementary school. With their guidance, I joined accelerated classes in middle school, ultimately graduating high school with college credits. My home wasn't an encouraging environment, so I feel incredibly fortunate to have had the support to discover my passion for science and pursue higher education. Looking back, I can recognize that this was an example of when an inclusive, fair, and equitable environment has benefited me. Experiences such as this helps me to recognize such behavior nowadays and do my part in fostering these environments.

During my freshman year at WMU, my parents divorced, and the divorce turned messy. In addition to being a full-time student and working part-time, I had to make time to appear in court as a witness, help my family clean and move out of our childhood home, and handle other family obligations. As my family and I distanced ourselves from my dad, we also parted with the last of the financial stability that he provided. This forced me into adulthood all at once and made it hard to prioritize my education. The effect of this period on my transcript is apparent, but also apparent is my resilience. When I emerged from this difficult period, I hit the ground running: I added a math major and a computer science minor to my degree, moved into an apartment near campus, began assisting with research (both the accelerator work and the semiconductor research, the latter of which I continue today), and co-founded Physics Club --- a club that often collaborated and held events with the university's Women in STEM and Society of Hispanic Professional Engineers clubs. I'm comforted having proven to myself that I can handle anything, whether that be trauma in my family and personal life or the 18-credit-hour semesters that followed. This same persistence and motivation guarantees my full effort towards succeeding at UC Boulder.

These experiences have driven my decision to pursue a PhD in physics at the University of Colorado Boulder. I am drawn to UC Boulder's commitment to academic rigor, fostering diverse, underrepresented perspectives, and giving opportunity to those who need it most. These are values I've embraced throughout my journey, and values that have previously enabled me to get to where I am today. I look forward to building on my research foundation and becoming part of a community that prioritizes growth, resilience, and inclusivity.

I'm eager to apply all that I've learned towards the fascinating research at \schoolabbr{}, and I look forward to the opportunity to learn from and contribute to your research teams. I am confident that I'd make a capable, committed researcher and condensed matter physicist. Thank you for your time and for considering my application.
\end{document}
