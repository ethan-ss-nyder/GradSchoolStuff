\documentclass[11pt]{article}

\usepackage{todonotes}
\newcommand{\school}{Michigan State University}
\newcommand{\schoolabbr}{MSU}
\newcommand{\dept}{Department of Physics and Astronomy}

\title{Academic Statement\\
	\large \dept{} at \school{}}

\author{Ethan Snyder}

\begin{document}
\maketitle
\subsubsection*{My Academic Experience}
All my life I had been certain that I wanted to study the stars. However, after a few years at university and over a year of invaluable research experience modeling semiconductor defects, I grew to be fascinated by the physics of the small and the physics governing the technology that powers the modern world. I learned that my deep appreciation for computers wasn't just essential to computational research, but computers could also be the subject of it. That motivated me to pursue condensed matter physics or quantum information science in graduate school. Graduate school is my next step, as it will allow me to further develop these interests, setting me up for my goal of researching semiconductors or quantum computing in the industry or as a professor. I see no better place to achieve this goal than \school{}, given the diversity and depth of research in these fields.

In my four years at Western Michigan University, I have been working towards Physics and Mathematics majors, and Computer Science and Astronomy minors. This wasn't always the goal, as I realized halfway through my undergraduate experience that I enjoyed learning about and working with computers too much to not encorporate that into my education. I also realized that I could benefit from a stronger background in mathematics, more than just the calculus that is required in physics. In my four years here at Western, I have certainly laid the foundations to be successful in computational research through both coursework and research experience.

Despite the difficulties in my personal life and the hefty courseload that accompanies two majors and two minors, I made the dean's list a couple times and most notably, in the spring of 2024, I recieved the Paul Rood Physics Scholarship from the Physics Department, acknowledging exceptional performance within the department. Over time, I have had the ability to improve my performance, understanding, and involvement with my education and research within the department, and you'll find that my transcript reflects that as well.

By far my most extensive research experience at Western Michigan University was my research work with semiconductors with Dr.\@ O'Hara of the Physics Department, which, at the time of writing, is being run on my account on MSU's computer cluster through the ICER program. The goal of this research is to calculate formation energies and model structural distortions in Ga and Al rich AlGaAs with a tellurium defect to corroborate and explain experimental findings. This research has familiarized me solid state physics, semiconductors, structural defects, density functional theory, VASP, and different functionals like PBE, HSE, and VCA. I've also learned countless things about how to interface with computer clusters, how to move files around, and how to analyze data. I've attached a digital poster I presented at the Lee Honors College at WMU as part of a 2024 summer scholarship.

In the spring of 2025, I will be beginning a semester project with Dr.\@ Korista of the Physics Department modeling stellar evolution in different classes of stars. This will provide me with a diversity of experiences with computational physics and data analysis across fields.
\todo{Add more to this, specifically about skills used and skills learned!}

During my time at Western, I've also become familiar with the Tandem Van de Graaff Accelerator in Rood Hall. While I have not operated it or officially participated in the research team that utilizes it, I have become familiar with how it is operated, what it is used for, and the data collection process, and have observed that research team often. As of this semester, I have also been spending time with the department's machine shop specialist learning to operate machines, machine shop etiquette, and how to communicate jobs effectively with machinists. I believe my familiarity with the lab setting and part manufacturing has made me a well-rounded physicist and well equipped to handle experimental research.

\subsubsection*{Why I choose \school{}}
Attending \schoolabbr{} would enable me to build upon my knowledge and rich experience with condensed matter physics, while also providing the opportunity to explore related topics, which \schoolabbr{} has a great diversity of . This would give me the opportunity and experience needed to go forward into a career, onto the frontier of research.

The Maghrebi, Dykman, Becker, and Pollanen Group stand out to me the most. I'm most interested in the Maghrebi group, as I'll have over a year of continuous experience in condensed matter research experience by the time I graduate undergrad, and I'm deeply interested in continuing to learn about condensed matter physics. I will have a solid foundational understanding to bring to his research and I'm eager to contribute.

Dr.\@ Dykman's work is also fascinating, particularly his work involving qubits and qubit frequency noise, as I'm interested in applications relevant to quantum computing and also have rich computational research experience to offer. His work on electrons in helium and nanomechanics is also fascinating. Overall, Dr.\@ Dykman's research seems to be a steep learning curve, but I know the skills I've learned in undergrad have equipped me to take on such large tasks.

Dr.\@ Becker's group also really appeals to my interest in quantum information science and quantum computing, and I believe my extensive background in solid-state physics, atomic dopants and defects, would provide me a great starting point in understanding and contributing to this team. While my experimental research experience is limited, 

The Pollanen group's work seems approachable to me, and I think strikes a good balance between being unfamiliar, understandable, and highly interesting. However, the extent of my experimental reserach experience is labs taken for credit and many hours around WMU's particle accelerator, although not explicitly operating it. I believe my familiarity with the lab setting and data analysis would enable me to learn from and contribute to the group's work.

There is no shortage of fascinating research being conducted at \schoolabbr{}, and I'm eager to apply all that I've learned towards great research such as Dr.\@ Maghrebi's non-equilibrium . Thank you for your time and for considering my application. I look forward to the opportunity to learn from and contribute to your research teams.
\end{document}
