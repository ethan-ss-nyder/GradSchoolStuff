\documentclass[11pt]{article}

\title{Academic Statement\\
\large Department of Physics and Astronomy at Michigan State University}

\author{Ethan Snyder}

\begin{document}
\maketitle
\subsubsection*{Why I choose \schoolabbr}
All my life I had been certain that I wanted to study the stars. However, after a few years at university and over a year of invaluable research experience modeling semiconductor defects, I grew to be fascinated by the physics of the small, and the physics governing the technology that powers the modern world. I learned that my deep appreciation for computers wasn't just essential to computational research, but that it is great motivation to pursue condensed matter physics or quantum information science in graduate school. Graduate school is my next step, as it will allow me to further develop these interests, setting me up for my goal of researching semiconductors or quantum computing in the industry or as a professor. I see no better place to achieve this goal than \school{}, given the diversity and depth of research in these fields.

There is no shortage of fascinating research being conducted at \schoolabbr{}, and for the sake of brevity I'm restraining myself from talking about every group and how interesting their work is. Thank you for your time and for considering my application. I look forward to the opportunity to contribute and learn from your research teams.

By far my most extensive research experience at WMU was my research work with semiconductors with Dr.\@ O'Hara of the Physics Department, which, at the time of writing, is being run on my account on MSU's computer cluster through the ICER program. The end goal of this research is to calculate formation energies, model structural distortions, and calculate the band gap in Ga and Al rich AlGaAs with a tellurium defect. This research has familiarized me with the deep acceptor defect in semiconductors, DFT, VASP, and different functionals like PBE, HSE, and VCA. I've also learned countless things about how to interface with computer clusters, how to move files around, and how to analyze data. I've attached a digital poster I presented at the Lee Honors College at WMU as part of a 2024 summer scholarship.

TALK ABOUT KORISTA WORK HERE.

Attending MSU would enable me to build upon my knowledge and experience with condensed matter physics, while also providing the opportunity to explore related topics. This would give me the opportunity and experience needed to go forward into a career, onto the frontier of research.

The Maghrebi, Dykman, and Pollanen Group stand out to me the most. I'd be able to contribute the most to the Maghrebi group, as I'll have over a year of continuous experience in condensed matter research experience by the time I graduate undergrad, and I'm deeply interested in that field. Looking through Dr.\@ Dykman's work, it seems like it would take me the longest to learn and contribute to, but I'd be very interesting to spend time learning the technicalities of his work and determining if it's for me. The Pollanen group's work seems the most approachable to me, and I think strikes a good balance between being unfamiliar, understandable, and highly interesting. Dr.\@ Becker's group also really appeals to my interest in quantum information science and quantum computing, and if admitted to MSU, I would be interested in learning more about their work.

TALK ABOUT OVERALL ACADEMIC RECORD --- HIGHLIGHT ACCOMPLISHMENTS AND AWARDS, EXPLANATIONS OF ACADEMIC RECORD.

\end{document}
