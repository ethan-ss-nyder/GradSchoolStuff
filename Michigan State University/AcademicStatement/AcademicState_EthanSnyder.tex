\documentclass[11pt]{article}

\usepackage{todonotes}
\newcommand{\school}{Michigan State University}
\newcommand{\schoolabbr}{MSU}
\newcommand{\dept}{Department of Physics and Astronomy}

\title{Academic Statement\\
	\large \dept{} at \school{}}

\author{Ethan Snyder}

\begin{document}
\maketitle
\subsubsection*{Why I choose \schoolabbr}
All my life I had been certain that I wanted to study the stars. However, after a few years at university and over a year of invaluable research experience modeling semiconductor defects, I grew to be fascinated by the physics of the small and the physics governing the technology that powers the modern world. I learned that my deep appreciation for computers wasn't just essential to computational research, but computers could also be the subject of it. That motivated me to pursue condensed matter physics or quantum information science in graduate school. Graduate school is my next step, as it will allow me to further develop these interests, setting me up for my goal of researching semiconductors or quantum computing in the industry or as a professor. I see no better place to achieve this goal than \school{}, given the diversity and depth of research in these fields.

By far my most extensive research experience at Western Michigan University was my research work with semiconductors with Dr.\@ O'Hara of the Physics Department, which, at the time of writing, is being run on my account on MSU's computer cluster through the ICER program. The goal of this research is to calculate formation energies and model structural distortions in Ga and Al rich AlGaAs with a tellurium defect to corroborate and explain experimental findings. This research has familiarized me solid state physics, semiconductors, structural defects, density functional theory, VASP, and different functionals like PBE, HSE, and VCA. I've also learned countless things about how to interface with computer clusters, how to move files around, and how to analyze data. I've attached a digital poster I presented at the Lee Honors College at WMU as part of a 2024 summer scholarship.

In the spring of 2025, I will be beginning a semester project with Dr.\@ Korista of the Physics Department modeling stellar evolution in different classes of stars. This will provide me with a diversity of experiences with computational physics and data analysis across fields.
\todo{Add more to this!}

Attending MSU would enable me to build upon my knowledge and experience with condensed matter physics, while also providing the opportunity to explore related topics. This would give me the opportunity and experience needed to go forward into a career, onto the frontier of research.

The Maghrebi, Dykman, and Pollanen Group stand out to me the most. I'm most interested in the Maghrebi group, as I'll have over a year of continuous experience in condensed matter research experience by the time I graduate undergrad, and I'm deeply interested in condensed matter physics. Looking through Dr.\@ Dykman's work, it seems like it would take me the longest to learn and contribute to, but I'd be very interested in learning the technicalities of his work and determining if it's for me. The Pollanen group's work seems the most approachable to me, and I think strikes a good balance between being unfamiliar, understandable, and highly interesting. However, the extent of my experimental reserach experience is labs taken for credit and many hours around WMU's particle accelerator, although not explicitly operating it. Dr.\@ Becker's group also really appeals to my interest in quantum information science and quantum computing, and if admitted to MSU, I would be interested in learning more about their work, although I have very experience with that field.

Paragraph here.
\todo{Talk about overall academic record. Highlight accomplishments, awards, and explanations of academic record.}

There is no shortage of fascinating research being conducted at \schoolabbr{}, and I'm eager to apply all that I've learned towards great research such as Dr.\@ Maghrebi's non-equilibrium . Thank you for your time and for considering my application. I look forward to the opportunity to learn from and contribute to your research teams.
\end{document}
