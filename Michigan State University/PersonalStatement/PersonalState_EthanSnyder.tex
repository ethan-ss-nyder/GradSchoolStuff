\documentclass[11pt]{article}

\newcommand{\school}{Michigan State University}
\newcommand{\schoolabbr}{MSU}
\newcommand{\dept}{Department of Physics and Astronomy}

\title{Personal Statement\\
	\large \dept{} at \school{}}

\author{Ethan Snyder}
\date{}

\begin{document}
\maketitle
\subsubsection*{Making a community at WMU}
My freshman year at WMU was isolated. The world was recovering from COVID, most of my classes were online, and for the classes that weren't, I commuted 40 minutes each way and lived at home. I was detached from WMU's community and student body, and was distracted by pressing, life-changing events happening within my family.

Sophomore year, I moved out to live off campus. I made a friend who was a fellow physics major, and him and I would study together. After some time, we attempted to gather the other physics majors in our cohort to create a stronger support system.

In that process, him and I discovered the remnants of Physics Club. Like many other clubs at WMU, it dwindled and died as COVID kept students off campus. We made friends with the seniors who were once members who still utilized the physics major commons. They were never officially made officers as the old officers graduated, and Physics Club was not registered as an RSO.

My friend and I spent the following summer cleaning the physics commons. We cleaned and organized the old papers that covered all tabletops. We organized textbooks on shelves. We dumped and emptied pop cans so old that the syrup had completely separated from the water. We vacuumed and put the couches on one side of the room, the tables and chairs on the other.

In the fall of 2023, him and I registered Physics Club as an official RSO, opened a bank account in it's name, and I begun attending biweekly meetings to ensure we had access to funding if we were to want to host events. We agreed that as co-founders, we would be co-presidents, and I assumed the extra tasks of being treasurer. We got the email list of all physics majors and begun outreach because Physics Club had no members.

We were initially unsuccessful in finding a community of those who were taking the classes we once took, and we were not able to reach those who had a general interest in physics. A few underclassmen trickled in over time and we introduced them to the commons and hosted study sessions if they wanted tutoring through courses. Still, we struggled to gather a community. I wanted to provide the community that I needed freshman year.

It wasn't until 2024's solar eclipse that Physics Club gained traction. I called parks, cities, and universities in the path of totality and asked to share their land for a watch party. I did the paperwork to ensure that the university would fund a bus to bring as many students as possible. I reached out to the Women in Stem RSO and we made the event a collaboration, and since then have partnered in many events. In the end, we were able to advertise enough to fill the bus, and it was an amazing experience, and an immensely rewarding endeavor despite the stress of planning.

Fall 2024, the semester in which I am writing this, has seen a boom in Physics Club members, and we're now in the process of partnering with Women in Stem and the Society of Hispanic Professional Engineers to plan a tour of Fermilab. Through in-person and email outreach, the physics commons can be somewhat loud during the day now. The work my friend and I have put in to make and maintain Physics Club has resulted in a strong community of students with a common passion for physics.

The large table in the center of the physics commons, historically, has been empty except for when my friend and I studied together. Now we find every chair around the table full with students studying, doing homework, and talking. Being able to walk into the commons and help a group of underclassmen with courses I've already taken is very rewarding (and good practice). It was the vision my friend and I had when we rearranged the room and put the tables together, and it's the community that I went searching for years ago.
%%%%%%%%%%%%%%%%%%%%%%%%%%%%%%%%%%%%%%%%%%%%%%%%%%%%%%%%%%%%%%%%%%%%%%%%%%%%%%%%%%%%%%%%%%%%%%%%%%%%%%%
\subsubsection*{Adversity}
I am a first generation college student. I grew up as the youngest of four boys in a single income household, always hovering just above the poverty line, and I am the only of the brothers to attempt a college degree. My mom has always said \textit{I don't know who you got that brain from}, but I figure I got it from her. She never had the chance to commit four years of her life to college. 

At the tail end of high school, my parents told me they were getting divorced. I was not shocked or overly distraught --- that is until it was discovered that my dad had been victim to a gambling addiction for over a decade. This discovery brought me to fear the fragility of finances and was a large reason for my pivot from astrophysics to condensed matter physics --- since that moment, I have had no safety net, and not only am I investing in my own education, but myself and my mom's future, who is cosigned on my student debt.

The divorce, previously calm, had lost it's civility, and I spent all of my freshman year, some of my sophomore year, appearing in court as a witness against my dad. While I was adjusting to college life, I found that I didn't have the time or energy to be present on campus. In that span of time --- freshman, sophomore year --- I had to make significant change to my life and all at once, transition from teenager to adult. The peak of this stress was the fall of 2022, when my family and I had to move out of my childhood home during the same week that I had midterm exams.

I can see the effect of this time period on my transcript, but also apparent is my resiliency. I got through this period with a deep understanding of the fundamentals to excel in later courses. Midway through my sophomore year, my family issues resolved and I finally felt it was appropriate to move to an off campus apartment and get serious about my education. I declared a math major on top of my physics major, and a computer science minor in addition to my astronomy minor. The next semester, I began working with Dr. O'Hara of WMU's Physics Department, and continue that same research to this day.

While neither myself nor my family have fully recovered from this event, I'm proud that I made it through, and with the weight of supporting my mom on my shoulders, I am determined to continue on.

\end{document}
