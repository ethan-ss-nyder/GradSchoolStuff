\documentclass[11pt]{article}

\newcommand{\school}{Michigan State University}
\newcommand{\schoolabbr}{MSU}
\newcommand{\dept}{Department of Physics and Astronomy}

\title{Personal Statement\\
	\large \dept{} at \school{}}

\author{Ethan Snyder}
\date{}

\begin{document}
\maketitle
\subsubsection*{Beyond Physics}
I've lived in Southwest Michigan my whole life, in the gap of light pollution between Grand Rapids and Kalamazoo. As a child, my gateway into science was astronomy, and through that dark patch of sky right above my head, I looked at the stars and decided science 

\subsubsection*{Making a community at WMU}
When I first started undergrad at Western Michigan University, the world was recovering from COVID, and for the few in-person courses I had, I commuted 30 minutes from home. I felt detached and isolated.

Sophomore year, I moved out to live off campus. I made a friend who was a fellow physics major, and he and I would study together. We sought out other physics majors to study together and to make a support system, but found that other physics majors were as detached as I had been. We also found that WMU's Physics Club had not been active in years.

We spent the following summer deep-cleaning the physics commons, and in the fall of 2023, him and I registered Physics Club as an official RSO, opened a bank account in its name, and I begun attending biweekly meetings to ensure we had access to funding if we were to want to host events. We agreed that as co-founders, we would be co-presidents, and I assumed the extra tasks of being treasurer.

It wasn't until 2024's solar eclipse trip that Physics Club gained traction. I called parks, cities, and universities in the path of totality and asked to share their land for a watch party. I did the paperwork to ensure that the university would fund a bus to bring as many students as possible. I reached out to the Women in STEM RSO and they attended, and since then we have partnered in many events. In the end, we were able to advertise enough to fill the bus, and it was an amazing experience, and it really exercised my ability to organize, lead, and execute plans.

Fall 2024, the semester in which I am writing this, has seen a boom in Physics Club members, and we're now in the process of partnering with Women in STEM and the Society of Hispanic Professional Engineers to plan a tour of Fermilab. I'm also serving as a link between the department faculty and the lowerclassmen, introducing undergraduates to research opportunities. As the most senior member of Physics Club, I also provide tutoring and guidance to the lowerclassmen through courses I've already taken, strenghtening my understanding of old course material.

The physics commons can now be somewhat loud during the day. We now have members from these RSO's we've collaborated with, and the diversity of students, voices, and backgrounds in the physics commons is refreshing in contrast to the small city in the corn fields where I grew up. I'm proud to have been a part of founding Physics Club and providing a supportive, safe space for all those interested in physics. It's immensely rewarding to finally be a part of that community that I had sought out when I was a freshman.
%%%%%%%%%%%%%%%%%%%%%%%%%%%%%%%%%%%%%%%%%%%%%%%%%%%%%%%%%%%%%%%%%%%%%%%%%%%%%%%%%%%%%%%%%%%%%%%%%%%%%%%
\subsubsection*{Persistance in the Face of Adversity}
I am a first generation college student. I grew up as the youngest of four boys in a single income household, always hovering just above the poverty line, and I am the only of the brothers to attempt a college degree. My mom has always said \textit{I don't know who you got that brain from}, but I figure I got it from her. She just never had the opportunity to commit four years of her life to college. 

At the tail end of high school, my parents told me they were getting divorced. It was the type of situation where I was not shocked or overly distraught --- that is until it was discovered that my dad had been victim to a gambling addiction for over a decade. This discovery brought me to fear the fragility of finances and was a large reason for my pivot from astrophysics to condensed matter physics. Since the divorce, I have had no safety net, and not only am I investing in my own education, but myself and my mom's future and stability.

The divorce, previously calm, had lost its civility, and I spent my early years of undergrad appearing in court. While I was adjusting to college life, I found that I didn't have the time or energy to be present on campus. In that span of time --- freshman, sophomore year --- I had to make significant change to my life and all at once transition from teenager to adult. The peak of this stress was the fall of 2022, when my family and I had to move out of my childhood home during the same week that I had midterm exams.

The effects of this time period are apparent on my transcript, but also apparent is my resiliency. I got through this period with a deep understanding of the fundamentals to excel in later courses. Midway through my sophomore year, my family issues resolved and I finally felt it was appropriate to move to an off campus apartment, and I got serious about my education. I declared a math major on top of my physics major, and a computer science minor in addition to my astronomy minor. The next semester, I began working with Dr. O'Hara of WMU's Physics Department, and continue that same research to this day.

While neither myself nor my family have fully recovered from this event, I'm proud that we made it through and I'm proud to have laid a strong foundation of knowledge during those times. Looking forward, I am determined to continue on, knowing that I can handle anything else life throws at me.

\end{document}
