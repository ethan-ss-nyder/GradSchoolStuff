\documentclass[11pt]{article}

\title{Personal Statement\\
	\large Department of Physics and Astronomy at Michigan State University}

\author{Ethan Snyder}

\begin{document}
\maketitle
\subsubsection*{Why I choose MSU}
My goal in life is to be on the academic frontier. I want to contribute to the world's library of knowledge. All my life I've taken from that, and no matter how small, I'd like to contribute.

Astronomy was my first love. I think for kids, it's the most approachable science other than paleontology. I'd argue that if you take plastic dinosaurs out of the equation, astronomy comes out on top. I grew up on Neil deGrasse Tyson and PBS Space Time, and looking back, I'm thankful that's what I chose to do, being the first generation of kids with unadulterated internet access. Around the age of 12, I asked to combine birthday and Christmas gifts and asked for a telescope. It was a refractor on a cheap equatorial mount that I didn't know how to use properly --- I used the lattitude knobs to aim up and down, and the R.A. knob to move side to side. I see those kinds of scopes sometimes referred to as "hobby killers" nowadays because they can be tough for beginners, but I'm thankful that I carried on. Every year until I was old enough to work, I asked to combine my birthday and Christmas to keep building that hobby.

In high school, it was requried for a course that I complete a job shadow. I decided instead of following my dad to work and having him sign my form, I would reach out to the physics departments of nearby colleges (in fact, Dr.\@ Dykman should be able to search my name and find my email from 2019). A professor at GVSU replied and told me to come spend a day in the department. He gave me the invaluable advice that, even if I were certain I wanted to pursue astrophysics, I should major in physics and leave my options open. I am very thankful for that advice.

Every year I spent at WMU, I grew less certain that I wanted to pursue astrophysics, and I began to explore other areas. On top of that, my parents had been in the process of divorcing, which revealed that my dad had a severe gambling addiction. I came face to face with the fragility of financial stability as my mom, my brothers, and myself distanced myself from my dad (for many reasons). For the first time in my life, I had no financial safety net and no illusion of one.

During my sophomore year, I decided to stop commuting, get an apartment near campus, pick up a math major, and add a computer science minor. This is when I began getting serious about my future. The investment I was making in my education was less so fueled by whimsy and dream, but by the need to be financially stable for myself and my mom, who has cosigned all my debt. Still, I wanted to find a balance.

So I pivoted. I found a research opportunity I did, and still do enjoy: computationally modeling semiconductors. I learned that wherever my future leads, I want to be working with computers --- I've always had a knack for coding, and I'd only gotten better with computers in undergrad, through this research and through classes. My knowledge in solid state physics now extends far beyond what has been taught in classrooms so far, and I'd love to continue building that knowledge in graduate school. However, I'm eager to explore areas like QCD and quantum information science, topics which have not been covered in detail in undergrad.

These interests led me to apply to graduate school at MSU. I've lived in Southwest Michigan my whole life, so MSU is local with many faculty whose research greatly interests me. The Maghrebi, Dykman, and Pollanen Group stand out to me the most. I'd be able to contribute the most to the Maghrebi group, as I'll have over a year of continuous experience in condensed matter research experience by the time I graduate undergrad, and I'm deeply interested in that field. Looking through Dr.\@ Dykman's work, it seems like it would take me the longest to learn and contribute to, but I'd be very interesting to spend time learning the technicalities of his work and determining if it's for me. The Pollanen group's work seems the most approachable to me, and I think strikes a good balance between being unfamiliar, understandable, and highly interesting. Dr.\@ Becker's group also really appeals to my interest in quantum information science and quantum computing, and if admitted to MSU, I would absolutely be interested in learning more about their work.

There is no shortage of fascinating research being conducted at MSU, and for the sake of brevity I'm restraining myself from talking about every group and how interesting their work is. Thank you for your time and for considering my application. I look forward to the opportunity to contribute and learn from your research teams.

%%%%%%%%%%%%%%%%%%%%%%%%%%%%%%%%%%%%%%%%%%%%%%%%%%%%%%%%%%%%%%%%%%%%%%%%%%%%%%%%%%%%%%%%%%%%%%%%%%%%%%%
\subsubsection*{Making a community at WMU}
My freshman year at WMU was isolated. The world was recovering from COVID, most of my classes were online, and for the classes that weren't, I commuted 40 minutes each way and lived at home. I was detached from WMU's community and student body, and was distracted by pressing, life-changing events happening within my family.

Sophomore year, I moved out to live off campus. I made a friend who was a fellow physics major, which are few and far between in the lower level physics courses a freshman or sophomore would take. Him and I would study together, and we attempted to gather the other physics majors in our cohort to create a stronger support system.

In that process, him and I discovered the remnants of Physics Club. Like many other clubs at WMU, it dwindled and died as COVID kept students off campus. We made friends with the seniors who were once members who still utilized the physics major commons. They were never officially made officers as the old officers graduated, and Physics Club was not registered as an RSO.

My friend and I spent the following summer cleaning the physics commons. We cleaned and organized the old papers that covered all tabletops. We organized textbooks on shelves. We dumped and emptied pop cans so old that the syrup had completely separated from the water. We vacuumed and put the couches on one side of the room, the tables and chairs on the other.

In the fall of 2023, him and I registered Physics Club as an official RSO, opened a bank account in it's name, and I begun attending biweekly meetings to ensure we had access to funding if we were to want to host events. We agreed that as co-founders, we would be co-presidents, and I assumed the extra tasks of being treasurer. Then we got the email list of all physics majors and begun outreach, because Physics Club had no members.

We were initially unsuccessful in finding a community of those who were taking the classes we once took, and we were not able to reach those who had a general interest in physics. A few underclassmen trickled in over time and we introduced them to the commons and hosted study sessions if they wanted tutoring through courses. Still, we struggled to gather a community. I wanted to provide the community that I needed freshman year.

It wasn't until 2024's solar eclipse that Physics Club begun to take off. I planned most of the trip, which included calling parks, cities, and universities in the path of totality and asking to use their land for a watch party. I also did the paperwork to ensure that the university would fund a bus to bring as many people as possible down to experience totality. I reached out to the Women in Stem club and the event turned into a collaboration, and thus the partnership and sharing of members between Physics Club and Women in Stem began. In the end, we were able to advertise enough to fill the bus, and it was an amazing experience, and an immensely rewarding endeavor despite the stress of planning.

Fall 2024, the semester in which I am writing this, has seen a boom in Physics Club members. Through in-person and email outreach, the physics commons can be somewhat loud during the day now. The work my friend and I have put in to make and maintain Physics Club and schedule regular astronomy outings and movie nights has resulted in a strong community of physics majors who support one another. The large table in the center of the physics commons, historically, has been empty except for when my friend and I studied together. Now we find every chair around the table full with physics majors and their identical textbooks open as they work through problems and support each other through classes. There's satisfaction in being able to walk into the commons and be able to help a group of underclassmen with courses I've taken. It was the vision my friend and I had when we rearranged the room and put the tables together, and it's the community that I went searching for years ago.
\end{document}
